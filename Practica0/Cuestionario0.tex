\documentclass[11pt]{article}

\usepackage[utf8]{inputenc}
\usepackage[spanish]{babel}
\usepackage{enumerate}
\usepackage{float}
\usepackage[hidelinks]{hyperref}
\usepackage[vmargin=2.3cm,hmargin=2.3cm]{geometry}


\title{MODELOS DE COMPUTACIÓN \\ CUESTIONARIO 0 }


\author{Elena María Gómez Ríos}


\begin{document}
\maketitle

\section{El Entscheidungsproblem.}
El Entscheidungsproblem (problema de decisión) fue un reto en lógica simbólica de encontrar el algoritmo general que decidiera si una fórmula de cálculo de primer orden es un teorema.

\section{Church, Turing y el Entscheidungsproblem.}
Alonzo Church y Alan Turing, de manera independiente, demostraron en 1936 que es imposible escribir el algoritmo de Entscheidungsproblem. Por lo tanto es imposible decir con un algoritmo si ciertas frases concretas de la aritmética son ciertas o falsas.

\section{Vida y contribución de Turing a la computabilidad.}

\section{Tesis de Church-Turing. Importancia.}

\section{Tesis de Church-Turing. Variantes y detractores.}


\end{document}