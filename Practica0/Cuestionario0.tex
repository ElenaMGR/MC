\documentclass[11pt]{article}

\usepackage[utf8]{inputenc}
\usepackage[spanish]{babel}
\usepackage{enumerate}
\usepackage{float}
\usepackage[hidelinks]{hyperref}
\usepackage[vmargin=2.3cm,hmargin=2.3cm]{geometry}


\title{MODELOS DE COMPUTACIÓN \\ CUESTIONARIO 0 }


\author{Elena María Gómez Ríos}


\begin{document}
\maketitle

\section{El Entscheidungsproblem.}
El Entscheidungsproblem (problema de decisión) fue un reto en lógica simbólica de encontrar el algoritmo general que decidiera si una fórmula de cálculo de primer orden es un teorema.

\section{Church, Turing y el Entscheidungsproblem.}
Alonzo Church y Alan Turing, de manera independiente, demostraron en 1936 que es imposible escribir el algoritmo de Entscheidungsproblem. Por lo tanto es imposible decir con un algoritmo si ciertas frases concretas de la aritmética son ciertas o falsas.

La noción de función efectivamente calculable que Church proponía era la de función $\lambda$-definible. Este concepto tiene su origen en un sistema lógico que en 1932-33 Church introdujo para fundamentar las matemáticas. Church empleó una notación formal que el denominó `cálculo lambda' para transformar todas las fórmulas matemáticas a una forma estándar. La demostración de teoremas se convierte así en una transformación de una cadena de símbolos en otra, en cálculo lambda, según un conjunto de reglas formales. Este sistema resultó ser inconsistente, pero la capacidad para expresar-calcular funciones numéricas como términos del sistema llamó pronto la atención de él y sus colaboradores. Así, Church habla en 1934 de la noción de función `efectivamente calculable' como función $\lambda$-definible. Además, bajo su tesis \textit{Tesis de Church} y utilizando la noción de función $\lambda$-definible, dio ejemplos de problemas de decisión irresolubles, y demostró que el Entscheidungsproblem era uno de esos problemas.

\section{Vida y contribución de Turing a la computabilidad.}

\section{Tesis de Church-Turing. Importancia.}

\section{Tesis de Church-Turing. Variantes y detractores.}


\end{document}