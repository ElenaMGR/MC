\documentclass[11pt]{article}

\usepackage[utf8]{inputenc}
\usepackage[spanish]{babel}
\usepackage{enumerate}
\usepackage{float}
\usepackage{amsmath}
\usepackage[hidelinks]{hyperref}
\usepackage[vmargin=2.3cm,hmargin=2.3cm]{geometry}
\usepackage{sectsty}
\sectionfont{\fontsize{12}{15}\selectfont}


\title{MODELOS DE COMPUTACIÓN \\ Practica 1. Introducción a la Computación. Lenguajes y Gramáticas }


\author{Elena María Gómez Ríos}


\begin{document}
\maketitle

\section{Describir el lenguaje generado por las siguientes gramáticas:}

\begin{enumerate}[a)]
	\item $S \rightarrow a$ $S_1$ $b$	\hspace{3cm}	$S_1 \rightarrow a$ $S_1$ $\vert$ $b$ $S_1$ $\vert$ $\varepsilon$

$L(G) = \lbrace aub$ $\vert$ $ u \in \lbrace a,b \rbrace^* \rbrace$	
	
	\item $S \rightarrow a$ $S$ $b$ $\vert$ $S_1$ \hspace{2,35cm} $S_1 \rightarrow \varepsilon$
	
$L(G) = \lbrace a^ib^i$ $\vert$ $ i \geq 0 \rbrace$	
	
	\item $S \rightarrow a$ $S$ $b$ $\vert$ $S_1$ \hspace{2,35cm} $S_1 \rightarrow c$ $\vert$ $\varepsilon$
	
$L(G) = \lbrace a^ic^jb^i$ $\vert$ $ i \geq 0, j \in \lbrace 0,1\rbrace \rbrace$		
	
	\item $S \rightarrow a$ $S$ $b$ $\vert$ $S_1$ \hspace{2,35cm} $S_1 \rightarrow c$ $S_1$ $d$ $\vert$ $\varepsilon$
	
$L(G) = \lbrace a^ic^jd^jb^i$ $\vert$ $ i,j \geq 0 \rbrace$	
	
	\item $S \rightarrow a$ $S$ $b$ $\vert$ $S_1$ \hspace{2,35cm} $S_1 \rightarrow a$ $S_1$ $\vert$ $b$ $S_1$ $\vert$ $\varepsilon$
	
$L(G) = \lbrace u$ $\vert$ $ u \in \lbrace a,b \rbrace^* \rbrace$	

\end{enumerate}

\section{Encontrar una gramática regular o una gramática libre de contexto que genere los siguientes lenguajes en el alfabeto A$=\lbrace$a,b,c$\rbrace$:}

\begin{enumerate}[$\bullet$]
	\item $u \in A^*$ si y solamente si verifica que $u$ empieza por el símbolo `a' y acaba con el símbolo `c'.

$S \rightarrow aS_1$	\hspace{3cm} $S_1 \rightarrow aS_1$ $\vert$ $bS_1$ $\vert$ $cS_1$ $\vert$ $c$
	
	\item $u \in A^*$ si y solamente si verifica que $u$ contiene un número par de símbolo `a'.

$S \rightarrow aS_1$ $\vert$ $bS$ $\vert$ $cS$ $\vert$ $\varepsilon$ \hspace{2cm} $S_1 \rightarrow aS$ $\vert$ $bS_1$ $\vert$ $cS_1$
	
	\item $u \in A^*$ si y solamente si verifica que $u$ tiene un número impar de símbolos y la letra central coincide con la última.
	
$S \rightarrow X$ $\vert$ $XS_1a$ $\vert$ $XS_2b$ $\vert$ $XS_3c$ \hspace{0,8cm} $S_1 \rightarrow XS_1X$ $\vert$ $a$ \hspace{0,8cm} $S_2 \rightarrow XS_2X$ $\vert$ $b$ \hspace{0,8cm} $S_3 \rightarrow XS_3X$ $\vert$ $c$	 \hspace{0,8cm} $X \rightarrow a$ $\vert$ $b$ $\vert$ $c$
	
	\item $u \in A^*$ si y solamente si verifica que $u$ no contiene la subcadena ab.
	
$S \rightarrow aS_1$	$\vert$ $bS$ $\vert$ $cS$ $\vert$ $\varepsilon$ \hspace{3cm} $S_1 \rightarrow aS_1$ $\vert$ $cS$ $\vert$ $\varepsilon$	
	
	\item $u \in A^*$ si y solamente si verifica que $u$ contiene 2 ó 3 símbolos c.
	
$S \rightarrow aS$	$\vert$ $bS$ $\vert$ $cS_1$ \hspace{2cm} $S_1 \rightarrow aS_1$ $\vert$ $bS_1$ $\vert$ $cS_2$ \hspace{2cm} $S_2 \rightarrow aS_2$ $\vert$ $bS_2$ $\vert$ $cS_3$ $\vert$ $\varepsilon$ \hspace{2cm} $S_3 \rightarrow aS_3$ $\vert$ $bS_3$ $\vert$ $\varepsilon$
	
\end{enumerate}

\section{Determinar si el lenguaje sobre el alfabeto A = \{a,b\} generado por la siguiente gramática es regular (justifica la respuesta):}

$S \rightarrow S_1bS_2$ \hspace{2cm} $S_1 \rightarrow aS_1$ $\vert$ $\varepsilon$ \hspace{2cm} $S_2 \rightarrow aS_2$ $\vert$ $bS_2$ $\vert$ $\varepsilon$\\

El lenguaje que genera esta gramática es el siguiente:\\

$L(G) = \lbrace a^ibu$ $\vert$ $i \geq 0, u \in \lbrace a,b \rbrace^* \rbrace$\\	

Como el lenguaje se puede generar mediante una gramática regular, el lenguaje es regular:\\

$S \rightarrow aS$	$\vert$ $bS_1$ \hspace{2cm} $S_1 \rightarrow aS_1$ $\vert$ $bS_1$ $\vert$  $\varepsilon$

\section{Identifique cuál de las siguientes afirmaciones es cierta con respecto a los lenguajes $L(G_1)$ y $L(G_2)$}

\hspace*{2.5cm}$ G_1 = \begin{cases} S \rightarrow X \\ S \rightarrow Y \\ X \rightarrow xXy 
  \\    Y \rightarrow xxYy \\ X \rightarrow \varepsilon \\ Y \rightarrow \varepsilon \end{cases}
\qquad \qquad
 G_2 = \begin{cases} S \rightarrow X \\ X \rightarrow Y \\ X \rightarrow xXy 
  \\    Y \rightarrow xxYy \\ X \rightarrow \varepsilon \\ Y \rightarrow \varepsilon \end{cases}
$
\\

Se puede observar que con el lenguaje $L(G_1)$ se genera el mismo número de $x$ que de $y$, ó el doble de $x$ que de $y$. En cambio con el lenguaje $L(G_2)$ se genera un número de $x$ igual o mayor al número de $y$. Por lo tanto la opción correcta es la a) $L(G_1)$ $\subset$ $L(G_2)$.

\end{document}